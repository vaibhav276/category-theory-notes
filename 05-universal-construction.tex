% Created 2020-01-29 Wed 15:17
% Intended LaTeX compiler: pdflatex
\documentclass[11pt]{article}
\usepackage[utf8]{inputenc}
\usepackage[T1]{fontenc}
\usepackage{graphicx}
\usepackage{grffile}
\usepackage{longtable}
\usepackage{wrapfig}
\usepackage{rotating}
\usepackage[normalem]{ulem}
\usepackage{amsmath}
\usepackage{textcomp}
\usepackage{amssymb}
\usepackage{capt-of}
\usepackage{hyperref}
\date{}
\title{Universal Constructions}
\hypersetup{
 pdfauthor={Vaibhav Pujari},
 pdftitle={Universal Constructions},
 pdfkeywords={},
 pdfsubject={},
 pdfcreator={Emacs 26.3 (Org mode 9.3)}, 
 pdflang={English}}
\begin{document}

\maketitle

\section{Terminal objects}
\label{sec:orgfe6dc0d}
Objects that have a unique morphism coming into them from every other object in the category

\begin{equation}
\forall c \in \zeta \exists! c \to T
\end{equation}

If there are more than one such objects, they are isomorphic

\subsection{Example}
\label{sec:orgc179554}
In a category of sets, the terminal object is a singleton set, because there can be only one unique function from any set to a singleton set.

\subsection{Code}
\label{sec:org9dd47c1}
\begin{verbatim}
data Terminal = MkTerm
bang :: a -> Terminal
bang _ = MkTerm
\end{verbatim}
In haskell, the object \texttt{()} is a terminal object

\section{Initial object}
\label{sec:org3548e7d}
Objects that have a unique morphism going out to every other object in the category

\begin{equation}
\forall c \in \zeta \exists! I \to c
\end{equation}

If there are more than one such objects, they are isomorphic

\subsection{Example}
\label{sec:org802f106}
In a category of sets, the initial object is a null set, because there can be
only one unique function from a null set to any other set.

\subsection{Code}
\label{sec:org9ea2a9b}
\begin{verbatim}
data Void
absurd :: Void -> a
absurd x = absurd x
\end{verbatim}

\section{Products}
\label{sec:org99e3acd}
A product of two objects \(a\) and \(b\) is an object \(a \times b\) that has projections \(\pi_1\) and
\(\pi_2\) to \(a\) and \(b\) respectively. These projections factorize any other
projections that exist from other objects \(c\) (potential products) to \(a\) and \(b\)

\subsection{Example}
\label{sec:org639505e}
A pair/tuple is a product

\subsection{Code}
\label{sec:org5e9fdb6}
\begin{verbatim}
data Pair a b = MkPair a b
fst :: Pair a b -> a
fst (MkPair a _) = a
snd :: Pair a b -> b
snd (MkPair _ b) = b

-- For potential products c
tuple :: (c->a, c->b) -> c -> Pair a b
tuple f g = \c -> MkPair (f c) (g c)

untuple :: (c -> Pair a b) -> (c->a, c->b)
untuple h = (fst . h, snd . h)

-- Bifunctor
-- Code easily obtained by visualizing that Pair a' b' is
-- a product of a' and b', and thinking of Pair a b as a potential
-- product c. So what we are basically doing is finding a mapping
-- from potential product to actual product
bimap :: (a->a', b->b') -> Pair a b -> Pair a' b'
bimap f g = tuple (f . fst) (g . snd)
\end{verbatim}

\section{Coproducts}
\label{sec:orga821aab}
If \(c,d \in Ob(\zeta)\), then the coproduct of \(c\) and \(d\) consists of:
\begin{enumerate}
\item An object \(c \oplus d \in Ob(\zeta)\)
\item A morphism \(Left: c \to c \oplus d\)
\item A morphism \(Right: d \to c \oplus d\)
\end{enumerate}
satisfying the property that \(\forall X \in Ob(\zeta)\), and morphisms \(L: c \to
X\), \(R: d \to X\) there exists a unique morphism \(c \oplus d \to X\), and the
diagram commutes.

\subsection{Code}
\label{sec:org4033af4}
\begin{verbatim}
data Either a b = Left a | Right b
-- Left :: a -> Either a b
-- Right :: b -> Either a b

either :: (c -> x, d -> x) -> Either c d -> x
either (f, g) (Left c) = f c
either (f, g) (Right d) = g d

-- Bifunctor
-- Code can be easily obtained by drawing a diagram
bimap :: (c->c', d->d') -> Either c d -> Either c' d'
bimap (f, g) = either (Left . f) (Right . g)
\end{verbatim}

\section{Exponentials}
\label{sec:org938c44b}
Given a function from a product \((a, b)\) to \(c\), we can always get a function
from \(a\) to a function \(f: b \to c\). A function \(f: b \to c\) is an exponential
object \(c^b\)

\subsection{Code}
\label{sec:org5c6e419}
\begin{verbatim}
curry :: ((a, b) -> c) -> a -> b -> c
curry f = \a -> (\b -> f (a,b))

-- And the counterpart
uncurry :: (a -> (b -> c) -> ((a, b) -> c)
uncurry f = \(a,b) -> (f a) b
\end{verbatim}
\end{document}
