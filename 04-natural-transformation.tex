\documentclass{article}
\author{Vaibhav Pujari}
\title{Natural Transformation}
\begin{document}
\section{Definition}

Given two functors, $F,G : \zeta_1 \to \zeta_2$ which map objects and morphisms
from $\zeta_1$ to $\zeta_2$, a natural transformation is a bunch of morphisms in
$\zeta_2$ that maps objects and morphism produced by $F$ into objects and
morphisms produced by $G$

  \textsf{\tiny
  \textbf{Notes}
  \begin{itemize}
  \item It is denoted by greek letters like $\alpha$ and really represents multiple
    morphisms under the hood, which live in $\zeta_2$
  \item Since there might not always be morphisms in $\zeta_2$ that can be
    utilized for natural transformation, sometimes there is no natural
    transformation available/possible.
  \end{itemize}
  }

\section{Constraints}

\begin{itemize}
\item \textbf{Diagram must commute}

  For each morphism in $\zeta_1$, there will be two morphisms in $\zeta_2$ (one
  due to $F$ and another due to $G$). These two morphism and the components
  (specific to this morphism) of a candidate natural transformation from $F$ to
  $G$ create a diagram. This diagram must commute for the candidate natural
  transformation to be a natural transformation

  \textsf{\tiny
  \textbf{Notes}
  \begin{itemize}
  \item One mistake I made initially while trying to understand natural
    transformations was that I assumed unconditional existence of natural
    transformations. Now I realize that the existence of natural transformation
    is subject to availability of morphisms in the target category ($\zeta_2$)
    which can be used as components to build a natural transformation. If no
    such morphisms exist, a natural transformation cannot exist.
  \item For programming it means if there is information loss due to
    abstraction, or the diagram does not commute, then a natural transformation
    might not exist between two functors.
  \end{itemize}
  }
\end{itemize}

\end{document}
