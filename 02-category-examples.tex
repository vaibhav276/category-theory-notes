\documentclass{article}
\author{Vaibhav Pujari}
\title{Categories}
\begin{document}
\section{Category of Vector Spaces}
\begin{itemize}
\item \textbf{Objects}

  Each vector space is an object

\item \textbf{Morphisms}

  Transformations (represented as matrices) between vector spaces

\item \textbf{Identities}

  The identity matrix for each vector space

\item \textbf{Composition}

  Matrix multiplication

\end{itemize}

\section{Monoid of Integers on Addition}
This is a singleton category - it contains only one object
\begin{itemize}
\item \textbf{Objects}

  $Integer$

  Note the abstraction here. We are not specifying which integer. So, for
  example both $20$ and $25$ are things which happen to be represented by the
  same object $Integer$ in this category.

\item \textbf{Morphisms}

  Set of integers, $Z$

  This is interesting. Here, the morphisms are just simple integers. So, for
  example to go from an object, say $20$ to an object, say $25$ (both are same
  objects because of our abstraction), we apply the morphism $5$. In case of a
  monoid category, we always start and end at the same object since there is
  only one object.

\item \textbf{Identities}

  The integer $0$ for our only object $Integer$

\item \textbf{Composition}

  Integer addition

  That is, to compose say two morphisms $4$ and $6$, we add them to get a third
  morphism $10$

\end{itemize}
\end{document}
