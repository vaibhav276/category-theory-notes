\documentclass{article}
\author{Vaibhav Pujari}
\title{Categories}
\begin{document}
\section{Category of Vector Spaces}
\begin{itemize}
\item \textbf{Objects}

  Each vector space is an object

\item \textbf{Morphisms}

  Transformations (represented as matrices) between vector spaces

\item \textbf{Identities}

  The identity matrix for each vector space

\item \textbf{Composition}

  Matrix multiplication

\end{itemize}

\section{Monoid of Integers on Addition}
This is a singleton category - it contains only one object
\begin{itemize}
\item \textbf{Objects}

  $Integer$

  Note the abstraction here. We are not specifying which integer. So, for
  example both $20$ and $25$ are things which happen to be represented by the
  same object $Integer$ in this category.

\item \textbf{Morphisms}

  Set of integers, $Z$

  This is interesting. Here, the morphisms are just simple integers. So, for
  example to go from an object, say $20$ to an object, say $25$ (both are same
  objects because of our abstraction), we apply the morphism $5$. In case of a
  monoid category, we always start and end at the same object since there is
  only one object.

\item \textbf{Identities}

  The integer $0$ for our only object $Integer$

\item \textbf{Composition}

  Integer addition

  That is, to compose say two morphisms $4$ and $6$, we add them to get a third
  morphism $10$

\end{itemize}


\section{Rubik's cube: Another Monoid}
\begin{itemize}
\item \textbf{Objects}

  Rubik's cube

  The abstraction here allows for any configuration of Rubik's cube to be
  represented by the singleton object

\item \textbf{Morphisms}

  An action on Rubik's cube is a morphism. We can see that its possible to
  combine two actions to produce a more complex action, and so on. In that way,
  from any configuration of Rubik's cube, there always exists a single
  (sufficiently complex) action, or morphism that solves the cube. Again a
  solved cube is no different than an unsolved cube as far as its significance
  in category is concerned

\item \textbf{Identities}

  A no-op action, which actually does not change the cube's configuration at
  all. We can think of it as maybe an action of touching the cube and putting it
  back. This is something I have performed many times :)

\item \textbf{Composition}

  Composition here is equivalent to sequencing the morphisms. I guess now its
  getting clearer that not always can we have a composition that forgets its parts

\end{itemize}

\section{Factors}
\begin{itemize}
\item \textbf{Objects}

  A set of natural numbers, $N$

\item \textbf{Morphisms}

  There is a single morphism between any two natural numbers $a$ and $b$ if $b$
  is a multiple of $a$

\item \textbf{Identities}

  Since every natural number is a multiple of itself (we get the same number
  when we multiply it by 1), there is an identity for each object in this category

\item \textbf{Composition}

  Composition here will simply represent a transitive relationship. If $a$ is a
  factor of $b$ and $b$ is a factor of $c$ then it implies that $a$ is a factor
  of $c$, which means there exists a morphism between $a$ and $c$

\end{itemize}
\end{document}
